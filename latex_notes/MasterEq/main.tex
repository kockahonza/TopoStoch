\documentclass[11pt]{article}
\usepackage[top=20mm,bottom=40mm,left=20mm,right=20mm]{geometry}
\usepackage[utf8]{inputenc}
\usepackage{parskip}

\usepackage{physics}
\usepackage{siunitx}
\usepackage{amsmath}
\usepackage[version=4]{mhchem}
\usepackage{graphics}
\usepackage{tikz}
\usetikzlibrary{math}

\usepackage{stackengine}
\usepackage{float}
\usepackage{tcolorbox}
\usepackage{hyperref}
\usepackage{cleveref}
\usepackage{lipsum}

\allowdisplaybreaks

\setlength{\parskip}{2ex}
\setlength{\parindent}{0em}

\newcommand\set[1]{\ensuremath{\{#1\}}}
\newcommand\textbff[1]{\textbf{\boldmath #1}}
\newcommand{\shortnote}[1]{\textit{\footnotesize (#1)}}

\stackMath
\newcommand{\suf}[2]{\stackunder[0.5pt]{\stackunder[1pt]{\ensuremath{#1}}{\rule{\widthof{\ensuremath{#2}}*\real{0.9}}{.1ex}}}{}}
\newcommand{\duf}[2]{\stackunder[0.5pt]{\stackunder[0.8pt]{\stackunder[1pt]{\ensuremath{#1}}{\rule{\widthof{\ensuremath{#2}}*\real{0.9}}{.1ex}}}{\rule{\widthof{\ensuremath{#2}}*\real{0.9}}{.1ex}}}{}}
\newcommand{\su}[1]{\suf{#1}{#1}}
\newcommand{\du}[1]{\duf{#1}{#1}}
\newcommand{\ssu}[1]{\scriptsize\su{#1}\normalsize}
\newcommand{\sdu}[1]{\scriptsize\du{#1}\normalsize}

\newcommand{\pp}{\ensuremath{\partial}}

\begin{document}
\begin{center}
	\LARGE
	\textbf{Notes on Master Equation Methods}
	\vspace{1em}
\end{center}

\section{Transition and Other Matrices}
Standard master equation has the form of
\begin{tcolorbox}
	\begin{equation}\label{eq:me}
		\partial_t p_i = \sum_j \qty(R_{ij}p_j - R_{ji}p_i)
	\end{equation}
\end{tcolorbox}
where $R_{ij}$ corresponds to the $j \rightarrow i$ transition.
For a steady state we must have $\partial_t p_i=0$ and if each term in the sum of \cref{eq:me} is independently 0 we have detailed balance.
In general however to solve for 0 it is convenient to put the equation in a eigenvalue like form such as
\begin{equation}\label{eq:dp}
	\partial_t p_i = \sum_j W_{ij}p_j
\end{equation}
for some matrix $\du{W}$.
This matrix can be expressed in terms of $\du{R}$ as
\begin{tcolorbox}
	\begin{equation}
		W_{ij} = R_{ij}-\delta_{ij}\sum_k R_{ki}
	\end{equation}
\end{tcolorbox}
as this leads to
\begin{align}
	\partial_t p_i & = \sum_j W_{ij}p_j = \sum_j \qty(R_{ij}-\delta_{ij}\sum_k R_{ki}) p_j = \\
	               & = \sum_j R_{ij}p_j - \sum_{j,k}\delta_{ij} R_{ki} p_j =                 \\
	               & = \sum_j R_{ij}p_j - \sum_{k} R_{ki} p_i =                              \\
	               & = \sum_j \qty(R_{ij}p_j - R_{ji} p_i) \qq{as required}
\end{align}

\section{Eigensystem of $\du{W}$}
Without proof, such a matrix will always have 0 as an eigenvalue.
If the matrix/system is connected this will be associated with one eigenstate.
Further, all other eigenvalues have negative real parts and their eigenstates' components sum to 0.

From \cref{eq:dp} it's clear that the eigenstate(s) with 0 eigenvalue form the steady state (or space).
The others are not so clear, consider having at $t=0$ a state $\su{p}(0) = \su{p}_0 + \su{\delta}$ with $\su{p}_0$ being a steady state.
Then we will have
\begin{equation}
	\partial_t \su{p} \eval_{t=0} = \du{W}\cdot\su{\delta} \qq{so that} \su{p}(\delta t) = \su{p}(0) + \delta t \du{W}\cdot\su{\delta} = \su{p}_0 + (\du{1}+\delta t\du{W})\cdot\su{\delta}
\end{equation}
This gets particularly interesting when $\su{\delta}$ is itself an eigenvalue of $\du{W}$.
If $\du{W}\cdot\su{\delta}=\lambda\su{\delta}$ then
\begin{align}
	\partial_t \su{p} \eval_{t=0} = \lambda\su{\delta} \qq{and} \su{p}(\delta t) = \su{p}_0 + (1+\delta t\lambda)\su{\delta}
\end{align}
which notably is of the same form as $\su{p}(0)$ just with a rescaled $\su{\delta}$.
Thus we can conclude that for any such starting state we have $\su{p}(t)=\su{p}_0+a(t)\su{\delta}$ for some function $a(t)$.
Notably here $\su{p}_0$ and $\su{\delta}$ do not have to be mutually orthogonal, if they are not than this is not the only way of writing the functional form of $\su{p}(t)$ but it is still a valid one.

This then leads to
\begin{equation}
	\su{p}(t+\delta t) = \su{p}_0 + a(t+\delta t)\su{\delta} = \su{p}(t)+\delta t \pdv{\su{p}}{t}\eval_t = \su{p}_0 + a(t)\su{\delta} + \delta t \lambda a(t) \su{\delta}
\end{equation}
after eliminating $\su{p}_0$ we get
\begin{align}
	a(t+\delta t)\su{\delta}            & = a(t)\su{\delta} + \delta t \lambda a(t) \su{\delta} \implies \\
	\frac{a(t+\delta t)-a(t)}{\delta t} & = \lambda a(t)
\end{align}
giving the predictable result of $a(t) = a(0)e^{\lambda t}$.

\begin{tcolorbox}
    So in summary, any non-steady eigenstate $\su{\delta}$ of $\du{W}$ can be seen as a perturbation mode which if superposed over a steady state will decay exponentially with its associated eigenvalue as $e^{\lambda t}$.
    Naturally if there are degenerate eigenstates with the same eigenvalue we get a perturbation space for that eigenvalue.
\end{tcolorbox}
though there remains the question of complex eigenvalues and eigenstates.
It seems natural that imaginary components of eigenvalues would correspond to oscillations and I mostly just hope that eigenstates do not have imaginary components

\section{Gillespie Simulations}\label{sec:gillespie}
Going off of the 2007 paper by Gillespie: \url{https://www.annualreviews.org/content/journals/10.1146/annurev.physchem.58.032806.104637}.
Alogirthm for realizing trajectories starts above equation 7.

Okay so from what I understand, the algorithm to choose the time and type of process out of a known set of processes each with its rate $r_i$ is as follows.
Choose a random process with the probability of getting each of proportional to it's rate (in other words equal to $\frac{r_i}{r_0}$).
Then choose a random number from an exponential distribution of $\text{pdf}(x) = e^{-r_0x}$ with $r_0=\sum_j r_j$ and we have $\tau$ equal to $r_i$ times the random number.

%section Gillespie Simulations (end)

\end{document}
