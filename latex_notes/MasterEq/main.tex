\documentclass[11pt]{article}
\usepackage[top=20mm,bottom=40mm,left=20mm,right=20mm]{geometry}
\usepackage[utf8]{inputenc}
\usepackage{parskip}

\usepackage{physics}
\usepackage{siunitx}
\usepackage{amsmath}
\usepackage[version=4]{mhchem}
\usepackage{graphics}
\usepackage{tikz}
\usetikzlibrary{math}

\usepackage{stackengine}
\usepackage{float}
\usepackage{cleveref}
\usepackage{lipsum}

\allowdisplaybreaks

\setlength{\parskip}{2ex}
\setlength{\parindent}{0em}

\newcommand\set[1]{\ensuremath{\{#1\}}}
\newcommand\textbff[1]{\textbf{\boldmath #1}}
\newcommand{\shortnote}[1]{\textit{\footnotesize (#1)}}

\stackMath
\newcommand{\suf}[2]{\stackunder[0.5pt]{\stackunder[1pt]{\ensuremath{#1}}{\rule{\widthof{\ensuremath{#2}}*\real{0.9}}{.1ex}}}{}}
\newcommand{\duf}[2]{\stackunder[0.5pt]{\stackunder[0.8pt]{\stackunder[1pt]{\ensuremath{#1}}{\rule{\widthof{\ensuremath{#2}}*\real{0.9}}{.1ex}}}{\rule{\widthof{\ensuremath{#2}}*\real{0.9}}{.1ex}}}{}}
\newcommand{\su}[1]{\suf{#1}{#1}}
\newcommand{\du}[1]{\duf{#1}{#1}}
\newcommand{\ssu}[1]{\scriptsize\su{#1}\normalsize}
\newcommand{\sdu}[1]{\scriptsize\du{#1}\normalsize}

\newcommand{\pp}{\ensuremath{\partial}}

\begin{document}
\begin{center}
    \LARGE
    \textbf{Notes on Master Equation Methods}
    \vspace{1em}
\end{center}

\section{Transition and Other Matrices}
Standard master equation has the form of
\begin{equation}\label{eq:me}
    \partial_t p_i = \sum_j \qty(R_{ij}p_j - R_{ji}p_i)
\end{equation}
where $R_{ij}$ corresponds to the $j \rightarrow i$ transition.
For a steady state we must have $\partial_t p_i=0$ and if each term in the sum of \cref{eq:me} is independently 0 we have detailed balance.
In general however to solve for 0 it is convenient to put the equation in a eigenvalue like form such as
\begin{equation}
    \partial_t p_i = \sum_j W_{ij}p_j
\end{equation}
for some matrix $\du{W}$.
This matrix can be expressed in terms of $\du{R}$ as
\begin{equation}
    W_{ij} = R_{ij}-\delta_{ij}\sum_k R_{ki}
\end{equation}
as this leads to
\begin{align}
    \partial_t p_i &= \sum_j W_{ij}p_j = \sum_j \qty(R_{ij}-\delta_{ij}\sum_k R_{ki}) p_j = \\
    &= \sum_j R_{ij}p_j - \sum_{j,k}\delta_{ij} R_{ki} p_j = \\
    &= \sum_j R_{ij}p_j - \sum_{k} R_{ki} p_i = \\
    &= \sum_j \qty(R_{ij}p_j - R_{ji} p_i) \qq{as required}
\end{align}

\end{document}
