\documentclass[11pt]{article}
\usepackage[top=20mm,bottom=30mm,left=20mm,right=20mm]{geometry}
\usepackage[utf8]{inputenc}
\usepackage{parskip}

% Mathy stuff
\usepackage{physics}
\usepackage{siunitx}
\usepackage{amsmath}
\usepackage[version=4]{mhchem}

% Visual stuff
\usepackage{graphics}
\usepackage{tikz}
\usetikzlibrary{math}
\usepackage{stackengine}
\usepackage{float}
\usepackage{tcolorbox}

% Misc
\usepackage{cleveref}
\usepackage{lipsum}

\allowdisplaybreaks

\setlength{\parskip}{2ex}
\setlength{\parindent}{0em}

\newcommand\set[1]{\ensuremath{\{#1\}}}
\newcommand\textbff[1]{\textbf{\boldmath #1}}
\newcommand{\shortnote}[1]{\textit{\footnotesize (#1)}}

\stackMath
\newcommand{\suf}[2]{\stackunder[0.5pt]{\stackunder[1pt]{\ensuremath{#1}}{\rule{\widthof{\ensuremath{#2}}*\real{0.9}}{.1ex}}}{}}
\newcommand{\duf}[2]{\stackunder[0.5pt]{\stackunder[0.8pt]{\stackunder[1pt]{\ensuremath{#1}}{\rule{\widthof{\ensuremath{#2}}*\real{0.9}}{.1ex}}}{\rule{\widthof{\ensuremath{#2}}*\real{0.9}}{.1ex}}}{}}
\newcommand{\su}[1]{\suf{#1}{#1}}
\newcommand{\du}[1]{\duf{#1}{#1}}
\newcommand{\ssu}[1]{\scriptsize\su{#1}\normalsize}
\newcommand{\sdu}[1]{\scriptsize\du{#1}\normalsize}

\newcommand{\pp}{\ensuremath{\partial}}

\begin{document}
\begin{center}
	\LARGE
	\textbf{Derivation of the formula relating chemical potentials and chemical concentrations}
	\vspace{1em}
\end{center}

We consider a lattice model with $\Omega$ sites at thermal equilibrium (canonical ensemble).
We denote the number of each species of ligand by $L_i$ and from statical physics we have that chemical potential of each is $\mu_i = \pdv{F}{L_i}$ with $F = U - TS$.
We have $U = \sum_i L_i \epsilon_i$ with the $\epsilon_i$ being the inherent chemical energies per molecule of each ligand type.
And finally we have $S=\si{k}\ln(W)$ with $W$ being the number of lattice microstates for any particular set of $L_i$ given by $W=\binom{\Omega}{L_1 \enspace L_2 \enspace \cdots}$.
Now we can calculate
\begin{align}
	\mu_i = \epsilon_i - \si{k}T \pdv{L_i} \ln(W)
\end{align}
but first approximate using Stirling's approximation
\begin{align}
	\ln(W) & = \ln(\binom{\Omega}{L_1 \enspace L_2 \enspace \cdots}) = \ln(\frac{\Omega!}{L_1!L_2!\cdots(\Omega-L_1-L_2-\cdots)!}) \\
	       & = \ln(\Omega!)-\ln(L_1!)-\ln(L_2!)-\cdots-\ln((\Omega-L_1-L_2-\cdots)!)
\end{align}
now assuming all the individual $L_i$ and $(\Omega-L_1-L_2-\cdots)$ are large we get to
\begin{align}
	\ln(W) \simeq & \enspace \Omega\ln(\Omega)-\Omega -L_1\ln(L_1)+L_1 -L_2\ln(L_2)+L_2 -\cdots                                       \\
	              & -(\Omega-L_1-L_2-\cdots)\ln((\Omega-L_1-L_2-\cdots)) + (\Omega-L_1-L_2-\cdots)                                    \\
	=             & \enspace \Omega\ln(\Omega) -L_1\ln(L_1) -L_2\ln(L_2) -\cdots -(\Omega-L_1-L_2-\cdots)\ln((\Omega-L_1-L_2-\cdots)) \\
	=             & \enspace \Omega\ln(\frac{\Omega}{\Omega-L_1-L_2-\cdots}) -L_1\ln(\frac{L_1}{\Omega-L_1-L_2-\cdots}) -\cdots       \\
	=             & \enspace -\Omega\ln(1-\frac{L_1+L_2+\cdots}{\Omega}) -L_1\ln(\frac{L_1}{\Omega-L_1-L_2-\cdots}) -\cdots           \\
\end{align}
now there's two alternative methods I want to try

\subsection{Approximate first}
Approximate by saying $\frac{L_1+L_2+\cdots}{\Omega}$ is small or in other words $\Omega \gg L_1+L_2+\cdots$ to get
\begin{align}
	\ln(W) \simeq & \enspace -\Omega\qty(-\frac{L_1+L_2+\cdots}{\Omega}) -L_1\ln(\frac{L_1}{\Omega}) -\cdots \\
	=             & \enspace L_1+L_2+\cdots -L_1\ln(\frac{L_1}{\Omega}) -\cdots                              \\
	=             & \enspace L_1\qty(1-\ln(\frac{L_1}{\Omega})) -\cdots                                      \\
\end{align}
this leads to
\begin{align}
	\mu_i = & \enspace \epsilon_i - \si{k}T \pdv{L_i} \ln(W)                                                                                                              \\
	\simeq  & \enspace \epsilon_i - \si{k}T\qty(1-\ln(\frac{L_i}{\Omega})+L_i\frac{-1}{\frac{L_i}{\Omega}}\frac{1}{\Omega}) = \epsilon_i + \si{k}T\ln(\frac{L_i}{\Omega})
\end{align}

\subsection{The full version}
Or just stick through with the pretty gross expressions to calculate
\begin{align}
	\pdv{L_i}\ln(W) = & \enspace \pdv{L_i} \qty(-\Omega\ln(1-\frac{L_1+L_2+\cdots}{\Omega}) -\sum_j L_j\ln(\frac{L_j}{K})) \qq{with} K=\Omega-L_1-L_2-\cdots \\
	=                 & \enspace -\Omega\frac{1}{1-\frac{\Omega-K}{\Omega}}(-\frac{1}{\Omega}) - \pdv{L_i}\sum_jL_j\ln(\frac{L_j}{K})                        \\
	=                 & \enspace \frac{\Omega}{\Omega-\Omega+K} - \sum_j\qty(\delta_{ij}\ln(\frac{L_j}{K}) + L_j\pdv{L_i}\ln(\frac{L_j}{K}))                 \\
	=                 & \enspace \frac{\Omega}{K} -\ln(\frac{L_i}{K}) -\sum_j L_j\frac{K}{L_j}\pdv{L_i}\frac{L_j}{K}                                         \\
	=                 & \enspace \frac{\Omega}{K} -\ln(\frac{L_i}{K}) -\sum_j K\qty(\frac{\delta_{ij}}{K}+L_j\frac{-1}{K^2}(-1))                             \\
	=                 & \enspace \frac{\Omega}{K} -\ln(\frac{L_i}{K}) -\sum_j \delta_{ij} + L_j\frac{1}{K}                                                   \\
	=                 & \enspace -1 +\frac{\Omega-\sum_jL_j}{K} -\ln(\frac{L_i}{K})                                                                          \\
	=                 & \enspace -\ln(\frac{L_i}{K})                                                                                                         \\
\end{align}
which is beautiful as it agrees perfectly and gives a simple expression for
\begin{align}
	\mu_i = \epsilon_i - \si{k}T \pdv{L_i} \ln(W) = \epsilon_i + \si{k}T \ln(\frac{L_i}{\Omega-L_1-L_2-\cdots})
\end{align}

\subsection{Relation to concentration}
In our simplified lattice model it seems clear that $\frac{L_i}{\Omega}$ is some measure of a concentration.
I'm not entirely sure how to relate it to a "real" experimental concentration but I'm sure it's possible, maybe worth doing sometime.

\end{document}
