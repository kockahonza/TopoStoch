\documentclass[aps, prx, letter, superscriptaddress, amsfonts, amssymb, amsmath, reprint, showkeys, nofootinbib, twoside, onecolumn]{revtex4-2}
\usepackage[english]{babel}
\usepackage[sort&compress]{natbib}
\usepackage[utf8]{inputenc}
\usepackage[colorinlistoftodos, color=green!40, prependcaption]{todonotes}
\usepackage{amsthm}
\usepackage{mathtools}
\usepackage{physics}
\usepackage[version=4]{mhchem}
\usepackage{xcolor}
\usepackage{graphicx}
%\usepackage[left=23mm,right=13mm,top=35mm,columnsep=15pt]{geometry} 
\usepackage{adjustbox}
\usepackage{placeins}
\usepackage[T1]{fontenc}
\usepackage{lipsum}
\usepackage{csquotes}
\usepackage{booktabs}
\usepackage[hypertexnames=false]{hyperref}
%\setlength{\marginparwidth}{2.5cm}
\hypersetup{colorlinks=true,citecolor=blue,linkcolor=blue,urlcolor=black}

\newcommand{\am}{\textcolor{red}}

% fixing the problem of Package babel Error: You haven't defined the language 'en' yet.
\renewcommand{\selectlanguage}[1]{}

\def\cor{\color{black}}
\def\cof{\color{black}}
\def\cob{\color{orange}}

% Kabir's shortcuts
\def\be{\begin{equation}}
\def\ee{\end{equation}}
\def\bmu{\begin{multline}}
\def\bea{\begin{eqnarray}}
\def\eea{\end{eqnarray}}
\def\n{{\bf n}}
\def\J{{\bf J}}
\def\p{\partial}
\def\nn{\nonumber}
\def\f{\frac}
\def\cor{\color{red}}
\def\cob{\color{blue}}
\newcommand{\edit}[1]{\textcolor{red}{{#1}}}
\def\l{\left(}
\def\r{\right)}
\def\fou{\widetilde}
\def\grad{\vec{\nabla}}
\def\rhat{\hat{\mathbf{r}}}
\def\that{\hat{\mathbf{e}}_{\theta}}
\newcommand{\iu}{{i\mkern1mu}}
\newcommand{\overbar}[1]{\mkern 1.5mu\overline{\mkern-1.5mu#1\mkern-1.5mu}\mkern 1.5mu}
\newcommand{\me}{\mathrm{e}}
\def\mbf{\mathbf}

\newcommand{\Lagr}{\mathcal{L}}

\begin{document}
 
\title{Non-equilibrium Protein Complexes as Molecular Automata}
\author{Jan Kocka}
\address{Department of Physics and Astronomy, University College London, London WC1E 6BT, United Kingdom}
\author{Kabir Husain}
\thanks{These authors contributed equally. Email: kabir.husain@ucl.ac.uk, j.agudo-canalejo@ucl.ac.uk}
%\email{kabir.husain@ucl.ac.uk}
\address{Department of Physics and Astronomy, University College London, London WC1E 6BT, United Kingdom}
\address{Laboratory for Molecular Cell Biology, University College London, London WC1E 6BT, United Kingdom}
\author{Jaime Agudo-Canalejo}
\thanks{These authors contributed equally. Email: kabir.husain@ucl.ac.uk, j.agudo-canalejo@ucl.ac.uk}
%\email{j.agudo-canalejo@ucl.ac.uk}
\address{Department of Physics and Astronomy, University College London, London WC1E 6BT, United Kingdom}


\begin{abstract}
	% Intro
	Considerable effort is being put into understanding how biological and biomimetic systems store and process information, a field known as molecular computation.
	Here we build a thermodynamically consistent kinetic model of a molecular complex made of identical subunits which can be in one of two states (e.g., phosphorylated or not).
	We then analyse the dynamics of the complex when each subunit can be modified by driven enzymes which act conditionally based on the state of its neighbours.
	For strongly driven enzymes we identify a one-to-one mapping to elementary cellular automata rules, each rule corresponding to a set of up to eight enzymes.
	Among these rules we find a rich set of behaviours, including multistability and dynamical steady states.
	Finally, we show how to deterministically manipulate the state of the complex by sequentially changing the rule (i.e. which enzymes are present) and map out the error-correcting capabilities of different steady states.

	% We map out the error-correcting capabilities of these steady states 
	% We identify the stability of the steady states up to random errors.

	% If we then consider changing the rule (corresponding to changing which enzymes are present) we can deterministically manipulate the complex between certain states to varying degrees based on the size and symmetry of the complex.

	% We can also identify error-correcting rules which are the most stable subject to random changes and identify their limits.


	% Understanding how cells make decisions in response to stimuli is a major challenge in biology today.
	% \ldots but it is still not understood at the level of information processing.
	% Considerable effort is also being put into devising synthetic biological systems which can process information, a field known as molecular computation.
	% In both fields DNA plays a crucial role facilitating memory.
	% % Both fields are focused on DNA being the memory which stores information with molecular machines operating on it.
	% However, this is not necessarily the case, we look at the information storage capability of an out-of-equilibrium allosteric complex.

	% NOTE: Want to get accross: we look at the simplest/minimal systems of which there are a discrete count of, we find a one-to-one mapping to ECA rule (which we use to identify symmetries)
	% NOTE: Maybe mention how they differ too? - stochastic and one change at a time


	% We identify a rich set of behaviours including multistability and dynamical steady states in various geometries.
	% We further draw an analoogy of each of the 256 possible minimal rules to a corresponding cellular automata rule and highlight the difference.
	% We identify a one to one map between such systems and elementary cellular automata rules and point out the differences.

	% We study the behaviour of a complex made of identical subunits, each in one of two states (e.g., structural conformations, phosphorylation).
	% We build a thermodynamically consistent kinetic model where the rate of changing the state of any one monomer can depend on the states of its neighbours.

	% Results/what we find from the model
	% We identify a connection between any such system and a cellular automata rule

	% Here we study the behaviour of a molecular complex made of identical subunits capable of changing between two states (e.g., structural conformations).
	% Here we study the behaviour of a molecular complex made of identical subunits, each allowed to change between two states, under the

	% Results/promises

	% % General introduction to topology in biology/biophysics - 98 words
	% Despite noisiness in the cellular environment, molecular systems show a high degree of robustness.
	% A recent new direction in understanding this apparent paradox is the study of topologically protected states in stochastic systems, which robustly confine the dynamics of the system to a lower-dimensional space.
	% However, it is unclear what the minimal biochemical ingredients are for such states to occur.
	%
	% % Getting into the project - ~110 words
	% Here, we study topological features in a non-equilibrium, thermodynamically-consistent model of a molecular assembly, made of subunits that undergo futile cycles of conformational change and phosphorylation.
	% When the subunits interact allosterically with each other, we find global, concerted cycles that emerge at the scale of the whole assembly.
	% These involve only a small subset of all possible conformations, analogous to topological edge currents in quantum systems.
	% We map out the kinetics, energetics, and biochemical interactions necessary to obtain distinct classes of topological behaviour.
	%
	% % Results-ish ->
	% Our results suggest that topological states can provide a minimal description of molecular coordination in protein complexes, such as circadian oscillators (e.g. KaiABC) or polymer assembly and disassembly (e.g. microtubules).
	% More broadly, our results demonstrate that stereotyped dynamics can arise purely from non-equilibrium kinetic effects, without the need for an underlying energy landscape to channel them.
\end{abstract}

\maketitle

% \section{Introduction}
%
% It is increasingly recognised that the route towards molecular computation is via multi-stable systems, with each stable state representing a distinct logical state or memory. In physical systems, these often correspond to multiple minima in the underlying energy landscape, as in (spin) glasses, self-folding origami, or multifarious self-assembly. A similar framework is invoked in biological systems (even beyond neuroscience), though these are often discussed in the language of attractors and fixed points in some underlying dynamical system: as in the Waddington landscape of development, or multistable fates in gene regulatory networks.
%
% Beyond natural systems, there is a growing interest in building synthetic, multistable circuits at the molecular scale. However, outside of a few instances, synthetic implementations are centred around DNA: either in conjunction with DNA-binding proteins, as in circuits made of synthetic transcription factors or site-specific DNA recombinases, or alone as in strand-displacement circuits assembled in-vitro. In contrast, natural biological signal processing often involves large, multi-protein complexes that `compute' via post-translational modifications. The design rules that underlie the computational capabilities of such protein circuits remain broadly unknown.
%
% In 1984, Francis Crick proposed a particular mechanism by which individual molecular complexes can harbour long-term memory despite stochasticity and molecular turn-over. The basis of his construction was a dimer of monomers that can each be activated (e.g., phosphorylated) by an enzyme. If, as he argued, the enzyme only acts on a monomer when the other monomer is already activated, then the dimer is capable of storing two distinct steady states: both monomers unactivated, or both activated. Crucially, as in von Neumann's error-correcting codes, the fully activated dimer is robust to stochastic perturbations: erroneous deactivation of one monomer is rapidly corrected by the enzyme. 
% % CamKII
%
% % Papers/Ideas:
%
% % Lee Adelman: Science (DNA computing with oligos)
% % Lisman 1985 + Crick 1984
% % Does each rule-set encode an error-correcting code? Robust to one-mutations.
%
% \section{Results}
%
% \subsection{Mapping molecular complexes to automata}
%
% We consider a complex made of $N$ identical monomers arranged in a cyclic polymer. Each monomer can take one of two states (representing, e.g., structural conformation or post-translational modifications such as phosphorylation), with the entire state $i$ of the complex denoted by a binary string, e.g. $i \in \{0000, 0001,..., 0101,... \}$ for $N=4$. The state of the complex changes when monomers transition between states, with rates $k_{i \to j}$. As each monomer makes physical contact with two neighbours, we allow these rates to depend on the states of neighbouring monomers.
%
% Similar models have been previously studied in the context of equilibrium allosteric complexes such as haemoglobin. There, the rates $k_{ij}$ are constrained to satisfy detailed balance,
% \begin{equation}    
% k^{\text{(allo)}}_{ij} \propto \exp(\beta(\epsilon_i - \epsilon_j)),
% \end{equation}
% \noindent where $\epsilon_i$ ($\epsilon_j$) is the energy of the complex before (after) the transition.
%
% In contrast, here we assume that there is no energetic preference for any of the states in the complex. Instead, the transitions are driven out of equilibrium by coupling to a (free) energy source such as a clamped reservoir of cytosolic ATP. Such a transition arises naturally when the transitions are enabled by an ATP-consuming enzyme (e.g., a kinase), or when the complex itself has ATP hydrolysis activity (as in AAA-ATPases such as the molecular clock KaiC).
%
% We consider the case in which each change of state is catalysed by a dedicated enzyme $\alpha$ whose activity depends on the state of the neighbouring monomers (e.g., a kinase that is recruited to a monomer only when both neighbours are phosphorylated). If all reactions couple to a single energy source, represented by an environmental reservoir clamped to a chemical potential $\Delta \mu > 0$, then the rate at which enzyme $\alpha$ catalyses the change of state $i \to j$ is
% % As such we build a kinetic model where each transition may take place as a result of a number of reaction mechanisms, each of which may be biased but must be reversible to maintain thermodynamic consistency.
% % These mechanisms then look as
% % \begin{equation}
% %     \text{enzyme } $\alpha$ + \cdots\overline{i0j}\cdots \ce{<=>[$r^{(\alpha)}_{\text{mech},ij,f}$][$r_{\text{mech},ij,b}$]} \cdots\overline{i1j}\cdots + \text{env}'
% % \end{equation}
% % \noindent in which the change of state of the environment is represented by the transformation $\text{env} \to \text{env}$ 
% % where if the transition is to be biased, the environment must have been affected in some way.
% % This can be encapsulated by the chemical potential $\mu_\text{mech}$ with the resulting transition rates being
% \begin{equation}
%     k^{(\alpha)}_{i \to j} = K_{\alpha} \frac{e^{\beta \Delta \mu}}{e^{\beta \Delta \mu} + 1},
% \end{equation}
% \noindent where $K_{\alpha}$ is the overall kinetic rate of the reaction, which includes the cellular concentration of enzyme $\alpha$. To be thermodynamically consistent, this same enzyme must also catalyse the reverse reaction with rate
% \begin{equation}
%     k^{(\alpha)}_{j \to i} = K_{\alpha} \frac{1}{e^{\beta \Delta \mu} + 1}.
% \end{equation}
% \indent As we have assumed that all states of the complex are energetically equivalent, these rates satisfy detailed balance only when $\Delta \mu = 0$. Note that while enzyme $\alpha$ preferentially drives the reaction $i \to j$, another enzyme $\beta$ might couple to the energy reservoir to drive the reverse reaction $j \to i$ (as in the famous `push-pull' regulatory motifs).
% % \begin{align}
% % 	 & r_{\text{mech},ij,f} = K_{\text{mech},ij} \frac{\exp(\beta\mu_\text{mech})}{\exp(\beta\mu_\text{mech})+1} \label{eq:rfmech} \\
% % 	 & r_{\text{mech},ij,b} = K_{\text{mech},ij} \frac{1}{\exp(\beta\mu_\text{mech})+1} \label{eq:rbmech}
% % \end{align}
% % where $K_{\text{mech},ij}$ is a matrix specifying how that mechanism depends on the affected monomer's neighbours and the overall rate of the reaction mechanism.
% % \indent In the limit of a strong non-equilibrium driving, $\Delta \mu \to \infty$, the model is specified entirely by two matrices $U$ and $D$.
%
% In practice, biological systems are strongly driven ($\beta\Delta \mu \gg 1$), such that reverse reactions do not occur on physiologically-relevant timescales. We therefore take the strongly-driven limit ($\Delta \mu  \to \infty$) and, for simplicity, suppose that each $K_{\alpha}$ takes on a constant non-zero value $K$ or is $0$ (i.e., the enzyme is present or absent). Later in the paper, we will relax both assumptions.
%
% In this limit, the dynamics of a complex is fully specified by the presence or absence of each enzyme type $\alpha$. There are eight possible enzymes: four `up' enzymes that catalyse the state change $0 \to 1$, and four `down' enzymes that catalyse the reverse reaction, depending on the possible states of the neighbouring monomers, shown schematically in Fig.~\ref{fig:1}.
% % numbers grouped into two $2 \times 2$ matrices, $K_\text{up}$ and $K_\text{down}$. 
% % To identify the most distinct behaviours we look at systems where all present mechanisms are equally likely to act ($K_{\text{mech},ij}$ only takes values of 0 or 1) and all are maximally driven $\mu_\text{mech} \rightarrow \pm\infty$.
% % Any such system can be fully specified by 8 numbers of 0 or 1 each specifying whether a particular transition of a monomer is possible based on its state and the state of its neighbours.
% We group these into two binary matrices $K^\text{up}$ and $K^\text{down}$, where, e.g., the element $K^\text{up}_{00}$ being 1 or 0 means that the transition
% $$\cdots 000 \cdots \rightarrow \cdots 010 \cdots$$ 
% \noindent can or cannot take place. Each pair of $K^\text{up}$ and $K^\text{down}$ specifies a Markov chain that operates on the state space of the protein complex.
%
% Notably, each of these systems can be thought of as analogous to a cellular automata rule.
% A cellular automata rule specifies the discrete time dynamics of a string of digits by prescribing to each digit which state it will take after the next step based on the states of itself and its neighbours.
% Molecular automata differ in that the transitions are stochastic, they happen one at a time and at random times.
% However, there is still a one to one map between the two and hence we will refer to the different systems of our model by their corresponding rule number. (*This should either include a diagram or reference a figure with a diagram?*)
%
% \subsection{Symmetries and classification}
% In addition to just having a correspondence, the two also follow the same symmetries.
% These are helpful to narrow down a search of the different rules as many of them will give the same behaviour subject to a changing the labels of the states.
% There are two symmetries of note, exchanging all 0s and 1s and reversing the complex/string.
% These form an Abelian group of symmetries the only other element being their combination and the identity.
%
%
%
%
% \begin{figure*}
%     \centering
%     % \includegraphics[scale=1]{FigureSmall/2024-05-08 Proofreading unified}
%     \caption{\textbf{Non-equilibrium transitions in an allosteric complex realise cellular automata rules.}   
%     }\label{fig:1}
% \end{figure*}
%
% \begin{figure*}
%     \centering
%     % \includegraphics[scale=1]{FigureSmall/2024-05-08 Proofreading unified}
%     \caption{\textbf{Conformations are funnelled into distinct stable attractors.}   
%     }\label{fig:2}
% \end{figure*}
%
% \begin{figure*}
%     \centering
%     % \includegraphics[scale=1]{FigureSmall/2024-05-08 Proofreading unified}
%     \caption{\textbf{Oscillations, branching cycles, and travelling domains in non-equilibrium attracting components.} 
%     }\label{fig:3}
% \end{figure*}
%
% \begin{figure*}
%     \centering
%     % \includegraphics[scale=1]{FigureSmall/2024-05-08 Proofreading unified}
%     \caption{\textbf{Finite driving leads to multi-stability and stochastic transitions.} ...\textbf{x} Bistable rule sets differ in their basins of attractions and therefore the dynamics of switching, allowing for sculptable thresholders. 
%     }\label{fig:4}
% \end{figure*}
%
\end{document}

